\documentclass{article}

\usepackage{fancyvrb}
\usepackage[T1]{fontenc}
\usepackage[utf8]{inputenc}

\title{Obligatorisk innlevering 1 hOsten 2014, INF3331}
\author{
  Dawood Ahmad - 
  \texttt{<dawood11@hotmail.com> (git bruker)}
  \and
  dawoodah - 
  \texttt{<dawoodah@student.matnat.uio.no>}
 }
\date{\today}

\begin{document}
\maketitle
\section*{Del 1: Bash}


\begin{Verbatim}
Dette gjelder alle oppgavene i Del 1:
\end{Verbatim}
Grunnen for at det er ../ i bash filene ved parameterne, er fordi oppgave filene ligger i mappa foran enn der filetree ligger.
For A kunne kjOre disse filene uten A endre pA noe, mA file\_tree ligge et "hakk" bak directory'et til oppgave filene.


\section*{Oppgave 1.1}

\begin{Verbatim}
Oppgaven lOst ved bruk av fOlgende komandoer:
find ../$1 -type f -mtime -$2 | xargs du | sort -n
\end{Verbatim}
- Find for A kunne finne (parameter 1 mappen)\hfill \break
- type f er for A kunne spesifisere at dette er en fil\hfill \break
- mtime (parameter2) sjekker alderen pA filen\hfill \break
- Den sjekker deretter fil stOrrelsen til filene med du og sorterer dem.

\subsection*{KjOring}

\begin{Verbatim}
1x-193-157-193-203:Oblig1 Dawood$ sh Oppgave1.1.sh file_tree 200
16	../file_tree/.DS_Store
72	../file_tree/Kq0Wv/MH/zWG/Exw0zNwi
72	../file_tree/Pkvye/htZiVgRE
80	../file_tree/Kq0Wv/MH/zWG/LwgfcJ8
96	../file_tree/Kq0Wv/5RYWI5kQ
208	../file_tree/Kq0Wv/MH/7GvTL2y
352	../file_tree/Kq0Wv/MH/zWG/1zeD9ON
368	../file_tree/zg/grYxji7
416	../file_tree/_CVcim
424	../file_tree/fJgme5F
464	../file_tree/Pkvye/vlfN/oHT/y2JjEL3A
800	../file_tree/LGPbdlRW
1072	../file_tree/Kq0Wv/MH/_Oj2c0QA
1128	../file_tree/zg/Hu/dNmOlK
1288	../file_tree/Kq0Wv/MH/XhdhBbk
1560	../file_tree/Kq0Wv/MH/Z9kP8NB
1560	../file_tree/Pkvye/vlfN/ZLbGhCmj
\end{Verbatim}

\section*{Oppgave 1.2}
\begin{Verbatim}
grep -r -l $2 ../$1
\end{Verbatim}
- grep printer ut de filene som inneholder ordet (parameter 2) fra filen (parameter 1)\hfill \break
- Pass pA at oppgave 1.2 er lOst fOr Oppgave 1.3, ellers vil ikke enkelte filer funnetfordi de ble slettet av Oppgave 1.3

\subsection*{KjOring}
\begin{Verbatim}
1x-193-157-193-203:Oblig1 Dawood$ sh Oppgave1.2.sh file_tree what
../file_tree/Pkvye/vlfN/ZLbGhCmj
\end{Verbatim}

\section*{Oppgave 1.3}
\begin{Verbatim}
echo deleting...\\
find ../$1 -type f -size +$2k\\
find ../$1 -type f -size +$2k -delete\\
echo deleted!!!
\end{Verbatim}
- Akkurat samme funksjon som oppgave 1.1, men her finner den fOrst alle filene som er stOrre enn 2 kilobyte. Deretter sletter den di funnede filene.\hfill \break
- Pass pA at oppgave 1.2 er lOst fOr Oppgave 1.3, ellers vil ikke enkelte filer funnet
fordi de ble slettet av Oppgave 1.3.


\subsection*{KjOring}

\begin{Verbatim}
1x-193-157-252-91:Oblig1 Dawood$ sh Oppgave1.3.sh file_tree 750
deleting...
../file_tree/Kq0Wv/MH/Z9kP8NB
../file_tree/Kq0Wv/MH/zWG/8puxfjS
../file_tree/Pkvye/vlfN/ZLbGhCmj
../file_tree/zg/Hu/vv/2KKnyIt5
deletet!!!
\end{Verbatim}

\section*{Oppgave 1.4}

\begin{Verbatim}
sort ../$1 -o $2
cat $2
\end{Verbatim}
Den tar inn parameter 1, som er den usorterte filen, sorterer den og deretter skriver det ut pA en ny fil og samtidig skriver ut resultatet til terminalen

\subsection*{KjOring}

\begin{Verbatim}
1x-193-157-252-91:Oblig1 Dawood$ sh Oppgave1.4.sh unsorted_fruits sorted_fruits
apple
grape
orange
pear
pineapple
\end{Verbatim}





\section*{Del 2: Python}

\begin{Verbatim}
1x-193-157-193-203:oblig01 Dawood$  python my_generate_file_tree.py file_tree 3 3
\end{Verbatim}
Jeg har tatt i bruk malen som var vedlagt pA kursets nettside.

I denne kildefilen har jeg fylt ut de oppgitte metodene for A kunne oppfylle de oppgitte kravene.
for A kunne generere tilfeldige strenger, har jeg benyttet meg av pakken Random.

\begin{Verbatim}
' rnd_str = prefix.join(random.choice(legal_chars) for _ in range(length))
return rnd_str '
\end{Verbatim}

For A kunne begrense hvor dypt dette treet skal gA og antall mapper som skal kunne opprettes
har jeg benyttet meg av den innebygde funksjonen random.randrange() fra pakken Random.

For at dette treet skal kunne opprettes rekursivt har jeg opprettet min egen metode:
\begin{Verbatim}
' def make_Dir_tree(target, dir_depth, dirs, verbose): '
\end{Verbatim}
Jeg har valgt A slett walk funksjonen, fordi det opprettes mapper og legges
til filer med umiddelbart, fOr det fortsettes til neste mappe.
metodene make\_Dir\_tree og populate\_tree gAr rekursivt inn i mappene og trekker fra dir\_depth eller antall files for iterasjon.

\end{document}



